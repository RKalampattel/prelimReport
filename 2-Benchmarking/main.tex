\documentclass[main.tex]{subfiles}

\begin{document}
\chapter{Benchmarking}
\chaplabel{benchmarking}

\section{Autonomous Platforms for Landmine Detection}

\section{Landmine Detection}
\textcolor{red}{Make sure to refer to this goal for this benchmarking type}
%• Implementation of autonomous detection and classification of subsurface objects, with a focus on identifying objects with a high likelihood of being a landmine. interpretation of the output signals from the GPR and metal detector will be processed with the aim to identify and confirm with a percentage the likelihood of a threat. Supplied data sets will be used to create and tune 2 a detection algorithm to meet this goal, and operational trials will be conducted to test the effectiveness of the developed system.

There have been similar projects that have used MD, GPR or a  combination of both to achieve detection and classification of landmines or unexploded ordinances. The detection and classification of landmines and subsurface objects are achieved through existing algorithms, such as, hough transformation, bayesian, feature extraction, kalman filters. 

\subsection{Metal Detectors}

\subsection{Ground Penetrating Radar}
\textcolor{red}{Need to relate to the scope as well, i.e. soil types and operational environments\\}
% What has been achieved with the GPR in terms of the operation environment and soil types. etc. that can be linked and based on
The detection of subsurface objects is one of the main uses for a GPR. This concept may be used in various applications including landmine detection. The GPR is able to detect landmines with various types of casing. An advantage of the GPR sensor is that there are both handheld and array types which assist in both manned and unmanned landmine detection operations. There are various types of commercial GPRs, such as NIITEK, SIRO-Pulse, and others. Most GPRs have provided with the 

In order detect and identify the subsurface objects, the GPR signals are required to be processed. There are various methods implemented to detect landmines using GPR \textcolor{red}{INSERT CITE}. The first part is the preprocessing of the signals in order to remove signal clutter. This is processed further in order to identify the object under the surface and minimise false positives.    
\textcolor{red}{NOTE: Need to mention anything commercialised here. Types of GPR used, and the operation frequency they used. The penetration depth etc. etc. Nothing too specific. Just how they've done it. Elaborate on the types in Literature review.} \textcolor{green}{perfect. this is exactly what i was going to put in here - i had a spreadsheet on the GoogleDrive with most of this data in it, but it seems to have gone walkabout}
%The following is in blue and will be modified soon-ish. Have to write an intro into it first. 
% Need to check everything

%Automated and unmanned landmine detection will be the main focus of the project,thus, attachment of the GPR to a platform is considered.  
%- NEED TO REWORD/REFER TO THE SIGNALS \\

%NOTE NO PHYSICAL MARKING OF THE LANDMINE

%It is also possible to detect the composition of the explosives used in the landmines, however, this is only used in conjunction with the electromagnetic properties returned to receive the signals. 

%Link to Challenges in identifying the landmine due to background noise etc. 
The current challenges with GPR sensors is the classification of the subsurface objects. %Need to mention that a database for identification is generally difficult to achieve with regards to current mines. Limitation include that the objects have similar reflected signals, such as rocks vs composite landmines. 

% Need to talk about the usefulness of the GPR and landmine together. How they complement each other. 


\subsection{Multisensor Systems}
% Link to challenges in integration 



\end{document}