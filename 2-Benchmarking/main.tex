\documentclass[main.tex]{subfiles}

\begin{document}
\chapter{Benchmarking}
\chaplabel{benchmarking}

\section{Autonomous Platforms for Landmine Detection}

\section{Landmine Detection}
\textcolor{red}{Make sure to refer to this goal for this benchmarking type}
%• Implementation of autonomous detection and classification of subsurface objects, with a focus on identifying objects with a high likelihood of being a landmine. interpretation of the output signals from the GPR and metal detector will be processed with the aim to identify and confirm with a percentage the likelihood of a threat. Supplied data sets will be used to create and tune 2 a detection algorithm to meet this goal, and operational trials will be conducted to test the effectiveness of the developed system.

There have been similar projects that may be researched in order to achieve the project's primary objective for landmine detection. The detection sensors that have been used are metal detectors, ground penetrating radar, or a combination of sensors.  

\subsection{Metal Detectors}

\subsection{Ground Penetrating Radar}
\textcolor{red}{Need to relate to the scope as well, i.e. soil types and operational environments\\}
% What has been achieved with the GPR in terms of the operation environment and soil types. etc. that can be linked and based on
The detection of subsurface objects is one of the main uses for a GPR. This concept may be used in various applications including landmine detection. The GPR is able to detect landmines with various types of casing. An advantage of the GPR sensor is that there are both handheld and array types which assist in both manned and unmanned landmine detection operations. 

In order detect and identify the subsurface objects, the GPR signals are required to be processed. There are various methods implemented to detect landmines using GPR\textcolor{red}{INSERT CITE}. The first part is the preprocessing of the signals in order to remove signal clutter. This is processed further in order to identify the object under the surface and minimise false positives.    

%The following is in blue and will be modified soon-ish. Have to write an intro into it first. 
\textcolor{blue}
{\paragraph{Preprocessing GPR data}
Prior to the identification process of the landmine, the scans from both the GPR and metal detector must be preprocessed in order to have a clear indication that an object is detected by the sensors. The main technique is background removal of the subsurface and clutter to accentuate the object detected \parencite{ko2012gpr,kruger2006new}. The signals are normalised in order to consider the difference in signals and the result is then processed further to identify the object \parencite{ko2012gpr}. Digital filtering techniques such as bandpass filters, Butterworth that are found in Matlab functions are also used \parencite{}.
\paragraph{Post-processing GPR data}
The methods in order to identify the scanned object after background removal for the GPR scans are kalman filters, pattern recognition and statistical learning algorithms. For metal detectors, there are pattern recognition techniques from the phase-plot data of the returned signal \parencite{long2006image}.
\subparagraph{Kalman Filter} The Kalman filter method defines a set of state matrices for the initial state and then changes these states accordingly \parencite{torrione2006ground}. The initial state matrix is defined when there is no target detected. The state matrix is then modified based on the assumptions that there is a detected target through the change in the reflected echo \parencite{torrione2006ground}. The detection then measures the error accordingly in order to find the relevant target \parencite{torrione2006ground}. 
%Check this section - 
\subparagraph{Pattern recognition} \parencite{pasolli2009automatic}
The pattern recognition process for classification of objects detected is monitored through the parabolic shapes of the landmine\parencite{pasolli2009automatic}. The signals from preprocessing are then processed through an edge detection algorithm in order to recognise the pattern\parencite{long2006image}. Certain features of the pattern of the parabola, such as the width and the sharpness at the peak, are recognised by using the relevant electromagnetic properties as discussed in \secref{gprcd}\parencite{long2006image}.The signals and properties are required to be compared to an existing range of data (database) \parencite{ko2012gpr}. This database is required to be built from various test sets \parencite{ko2012gpr}.\\
\subparagraph{Statistical Algorithms} Statistical learning algorithms are based similarly to fuzzy logic and decision-tree learning in artificial intelligence \parencite{sakaguchi2014physics}. This is based on the probability that the object is likely to be a landmine based on the percentage of certainty \parencite{sakaguchi2014physics}. The remaining percentage is the probability of this as another object or ground clutter \parencite{sakaguchi2014physics}. The variations in signals is dependent on the electromagnetic signal of the object as mentioned in \secref{gprcd}. Decision tree learning works by collecting a series of metrics about the scenario, in this case the outputs from the sensors, the features of the Fast Fourier Transform (FFT), signal to noise ratio (SNR) of key responses and paring these metrics with known outcomes of ‘mine’ or ‘not mine’ \parencite{sakaguchi2014physics}. Provided the software implementation is correct and the sensory data is precise enough for a possible distinction to be made, the software will identify the signature of a mine object. 
\nomenclature[A]{FFT}{Fast Fourier Transform}% 
\nomenclature[A]{SNR}{Signal to Noise Ratio}% 
}

%Automated and unmanned landmine detection will be the main focus of the project,thus, attachment of the GPR to a platform is considered.  
%- NEED TO REWORD/REFER TO THE SIGNALS \\

%NOTE NO PHYSICAL MARKING OF THE LANDMINE

%It is also possible to detect the composition of the explosives used in the landmines, however, this is only used in conjunction with the electromagnetic properties returned to receive the signals. 

%Link to Challenges in identifying the landmine due to background noise etc. 
The current challenges with GPR sensors is the classification of the subsurface objects. %Need to mention that a database for identification is generally difficult to achieve with regards to current mines. Limitation include that the objects have similar reflected signals, such as rocks vs composite landmines. 

% Need to talk about the usefulness of the GPR and landmine together. How they complement each other. 


\subsection{Multisensor Systems}
% Link to challenges in integration 



\end{document}