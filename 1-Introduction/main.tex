\documentclass[main.tex]{subfiles}

\begin{document}
\pagenumbering{arabic}	% Numbering style for body

\chapter{Introduction}
\chaplabel{introduction}

\section{Background and Motivation}
\seclabel{background}
\textcolor{green}{I really hate this opening sentence. I can't think of a better one but I think this is a pretty atrocious opening.}
\textcolor{purple}{I agree, its kind of a nothing sentence. maybe we could place it after the "one of the rists present...." sentence and introduce the 'future mission' as landmine detection rather than just an autonomous platform for 'applications' (what even is applications?).}
\textcolor{orange}{What about an extra line that actually introduces who the DSTG is or what they do? Something like ``The Defence Science and Technology Group (DSTG) is a government agency that applies science and technology to protect and defend Australia and it's national interests'' (can't use this actual line since it's been copied from \url{http://www.australia.gov.au/directories/australia/dstg}) - RK} \textcolor{green}{I like this, PD} 

A future mission for the Defence Science and Technology Group (DSTG) is to develop an autonomous platform \sout{vehicle} for Australian Defence Force (ADF) applications.\sout{, with the main focus being to reduce risk to ADF personnel.} 
\textcolor{red}{(Makes more sense to introduce the risk first, then say the DSTG is trying to reduce this risk - RK)} 
One of the risks present for ADF personnel in conflict zones, such as in the Middle East, are Improvised Explosive Devices (IEDs) and landmines, namely anti-tank (AT) and anti-personnel (AP) mines. Accurate detection of these threats, while reducing the risk to ADF personnel, \sout{before they detonate} is a major incentive for the development of an autonomous platform with landmine \sout{threat} detection capabilities.\sout{, namely landmine detection.}
\\
\nomenclature[A]{DSTG}{Defence Science and Technology Group}% 
\nomenclature[A]{ADF}{Australian Defence Force}% 
\nomenclature[A]{IED}{Improvised Explosive Device}%
\nomenclature[A]{AT}{Anti-Tank (Mine)}%
\nomenclature[A]{AP}{Anti-Personnel (Mine)}%

There are two primary methods of landmine detection currently used by the ADF, manual screening with the use of a metal detector (MD) and vehicle mounted screening using an MD and a Ground Penetrating Radar (GPR). Manual screening is the most common method, and is accurate when detecting metallic objects. However, it is slow and lacks the ability to detect composite based landmines. Vehicle mounted detection methods are fast, and are able to detect composite mines due to having a GPR, but they lack the ability to accurately determine the location of a threat. 
\textcolor{red}{(Do they lack this ability? - RK)} \textcolor{blue}{Re-word to say that it is more difficult to detect landmines accurately without human assistance. - RP}
\sout{Because of} Due to the use of large sensor arrays, and an inadequate level of \sout{lack of} sensor integration, threats are unable to be confirmed, requiring secondary manual sweeps of the area. Both manual and vehicle mounted methods require personnel to be in close proximity to mines and IEDs, posing a considerable risk to their lives. %\textcolor{blue}{There are many new technologies present for landmine detection however an integration of multiple systems is required to allow for threat detection as well as confirmation.} <- I don't like this last sentence-peter
\\
\nomenclature[A]{GPR}{Ground Penetrating Radar}%
\nomenclature[A]{MD}{Metal Detector}%

Through communications with the DSTG, it was apparent that the safety of ADF personnel  was a priority. Currently, there is no method for landmine detection as well as confirmation in a single sweep where the operator is removed from the system. Therefore, this project proposes the use of an autonomous and unmanned platform for the mounting of current ADF landmine detection technologies with the aim to reduce the risk imposed upon ADF personnel. The focus of the project will be on the development of an autonomous platform as well as multi sensor integration targeted for threat detection and confirmation through the use of a metal detector in conjunction with a DSTG loaned GPR.
\\

%the rest is deletable
%\\
%\\platform automation is a necessity when coupled with the task of landmine detection. This project proposes the use  of a autonomous and unmanned platform for mounting landmine detection technologies.
%\textcolor{green}{  \\
% I get what you are trying to say but it's not worded right. /need to put unmanned as a autonomous paltfrom could still have a dude on it-peter

%Therefore, further risk mitigation highlights the need for a system that is able to detect, confirm and accurately locate a threat. Thus, reducing the required field time to manually find and identify the threat. 
%Changed a few words, sound okay? = Rac
% does ^ belong in intro and motivation as well? then use another way to introduce the sentence below. 
%}

\textcolor{red}{do we need to discuss the issues and hardships for the project here?}
\textcolor{red}{need to expand more on the need and motivation}
\\\par



\section{Project Goal}
\seclabel{projectGoal}

The goal for this project is to develop a prototype for \textcolor{green}{autonomous} landmine detection and confirmation\textcolor{green}{, with a focus on reduction of false positive instances}. This will be completed through the use of an autonomous \textcolor{green}{vehicular} platform and compilation of \textcolor{green}{sensory} data from a MD and GPR array. \sout{This requires}\textcolor{green}{Achieving this goal will require} multi sensor integration and advanced signal processing as well as software development for the platform navigation and automation. The project objectives necessary to achieve this goal are \sout{highlighted} \textcolor{green}{described} below. 
%These objectives are labelled as primary or extended, depending on their difficulty and scope.




 
\subsection{Primary Objectives}
% OVER THE WHOLE PRIMARY OBJECTIVES SHOULD WE BE USING THE WORD 'WILL' OR 'SHOULD'? as in: this will/should be implemented...
\seclabel{primary}
The primary objectives are the minimum allowable outcomes for the project and should be delivered upon the completion of the project. These objectives are:
%i reckon get rid of 'for the project'
% yes i think so too
\textcolor{red}{do we need to discuss how we are going to achieve these objectives? \textbf{NO!, i think we should only include what will indicate completion of the objective } - I think we need to include something on how achievable it is (as per the project charter feedback)} \textcolor{green}{I 100 percent agree. outline of what we want to achieve, very brief statement on what we consider achieving to be. The rest of this section is pretty much on-point though.}
\begin{itemize}
\item Selection and subsequent modification of a pre-existing platform for remote control via a hand held device. Necessary control systems will be implemented within the platform. There will be emphasis on the throttle, steering and brake systems to allow for full system control in a safe manner. Sufficient control dexterity Will be tested for to mimic those of a manual operator. Each system will be tested separately to ensure that they function correctly and are safe to operate remotely. Once all systems are operating independently, simultaneous operation shall be conducted with the aim to be in control of the platform at all instances. Once the platfrom is able to be operated remotely, the objective will be achieved with the deliverable of a working platfrom.%A full range of motions will be performed to ensure that the performance is as desired.
% "A full range of motions will be performed to ensure that the performance is as desired" - a reader doesnt know what performance is required yet. i dont know if this is a real problem though, just thought id point it out.
% ^^ whoever wrote that comment - agreed

\item Modify a pre-existing platform to accommodate a GPR and a metal detector. Structural modifications will be conducted on the platform so that it can support and carry the chosen sensors safely while adhering to the operating requirements of both the vehicular platform and sensors. %The necessary support structures will be designed so as not to cause interference with the sensors that could impair the signal responses. 
% ^ is this sentence part of the goal? i feel like it needs to be more general, we can talk about interference later on down the track.
Tests with the sensors will be used to determine the construction material for the attachment brackets, as well as determining the placement of the sensors. Sensor location with respect to the platform should not interfere with the ideal operating conditions for the sensors. The objective will be achieved when the sensors are mounted to the platfrom without interfernce.

\item Implement software to function as the framework for automation of the vehicle, allowing the platform to autonomously navigate and collect data under supervision from an operator. 
The required navigation and control systems will be implemented once the platform is able to be directly controlled by a remote operator. The location, heading, speed, and status of the metal detector, GPR, and platform will be required to be known for the framework for automation to be considered complete.


\item Implementation of autonomous detection and classification of subsurface objects, with a focus on identifying objects with a high likelihood of being a landmine. Interpretation of the output signals from the GPR and metal detector will be processed with the aim to identify and confirm with a percentage the likelihood of a threat. % \textcolor{red}{this is too wordy from here} 
%with a focus on achieving complete detection of subsurface objects and determination of mines with a high ratio of correct determinations to false-positives.\textcolor{red}{to here} 
% how it is going to be achieved doesnt go in this section i dont think, "determination of mines with a high ratio of correct determinations to false-positives" is ok though, but too many determinations.
Data sets are required to create and tune a detection algorithm for both the metal detector and GPR. operational trials are needed to be conducted to test the effectiveness of the developed system. 
%Supplied data sets will be used to create and tune a detection algorithm to meet this goal, and operational trials will be conducted to test the effectiveness of the developed system.
%again, i dont think we need to include how things will be done, but "operational trials will be conducted to test the effectiveness of the developed system" is ok.
\item Logging of detected objects with GPS coordinates and determined confidence of the object being a landmine. The ability to trasnmitt a location to a hand helpd device would be required. GPS location of the platfrom is also required to achieve this goal.  %Once a landmine has been located, this data will be recorded and transmitted to a hand held device held by a supervising operator. This device will display a map of the area of interest with a location of the detected object on the map.
\end{itemize}

\subsection{Extended Objectives}
\seclabel{extended}
Extended objectives are those that are slightly outside the scope of the project and may be completed time permitted. These objectives are:

\begin{itemize}
\item Automatic traversal of a designated area enclosed by user defined way points. This is the first stage of automation. Initially, single way point navigation will be conducted, requiring the platform to travel from one predefined point to another before coming to a stop. Once this is achieved, way-points enclosing an area will be used requiring the platform to generate its own path to traverse the enclosed area.
\item Provide a live video feed as well as live updates of vehicle location and direction to the operator's handheld device, allowing for greater visualisation of operating conditions of the remote vehicle, as well as greater ability to detect and respond appropriately to unknown obstacles in the search area.
\item Physical marking of landmine locations by the remote platform at the time of detection, allowing mine-removal technicians to easily identify the location of interest with greater accuracy, and free from any potential calibration errors or systematic offsets from a GPS position.
\end{itemize}

\section{Project Definition}
\seclabel{projectDefinition}
Due to the large number of countries where ADF personnel are deployed \textcolor{green}{and are at risk from IEDs and mines}, the differing terrain and geographical properties present a challenge to build a single dedicated landmine detecting device. This section will cover the platforms operational scenarios, environments and performance specifications for different possible operating locations. \textcolor{green}{An u}nderstanding of the operating conditions of a delivered system will influence the conceptual design of the project solution.
\subsection{Scenario of Operation}
\seclabel{scenario}

\subsubsection{General Mission Profile}
\seclabel{MissionProfile}
The general mission profile describes the \sout{basic} \textcolor{green}{intended} use case of the platform regardless of the platform selected or geographical variables. This is the expected manner in which the delivered system will be used by an operator to achieve the project goal of autonomous landmine detection, and has formed the basis of hardware and software designs.
\begin{quote}\textit{The platform starts from its home location, and travels via remote control to an area of interest; the boundaries of this area are selected using GPS coordinates. Once it has arrived, the platform autonomously scans its surroundings for landmines using both a metal detector and GPR. Possible points where landmines are identified are logged with GPS coordinates allowing for a map of the area to be developed. On-board sensors are used for object detection (object avoidance is not expected to be implemented), and a video feed may be streamed in real time to the operator. Once the platform has scanned the designated area, it returns back to the starting location where it is manually controlled back to its home location. If at any point the operator wishes to abort the mission or signal between the platform and the tablet becomes lost, the platform will stop immediately.}
\end{quote}
\subsection{Operating Environments}
\seclabel{operatingEnvironments}
\subsubsection{Real World Environments}
\seclabel{IRL}


The real world operating conditions for the platform in different locations are considered, \sout{. \textcolor{green}{(So? this sentence doesn't add anything.)}} taking into consideration \sout{These considerations take into account} the types of vegetation, soil and terrain, as well as mine types and affected area sizes. The countries considered are those where ADF personnel are present and heavily affected by mines. These locations include Egypt and regions of the Middle East, namely Iran \parencite{AustralianGovernment2016}. The requirements and needs of mine removal operators in these environments are used to identify the needs and requirements for an effective mine detection system.

\subsubsection{Eqypt}
\seclabel{Egypt}
Egypt is listed as the country most contaminated by mines with an area of roughly 25,000 kilometers squared containing over 23 million buried anti tank and anti personal mines \parencite{Rushfan2008}. The main contaminated areas are along the North coast and the Sinai Peninsula. These areas represent 22\% of the total area of Egypt and consist of barren shifting sands and Lithosols with large flat expanses of sandy land \parencite{Nahrawy2011}. 
 The chosen platform would then be required to travel over sandy terrains and over small obstacles such as rocks with the sensors being able to detect landmines through sandy and loose soil. 
 \subsubsection{Iran}
 \seclabel{Iran}
 Iran has an area of approximately 18 thousand square kilometres of land contaminated with 12-16 million mines left over from the 1980-1988 Iran-Iraq conflict \parencite{landmineMonitor2015}. The main areas of contamination are along the provinces of Khuzestan which is mainly covered with sandy soils similar to those in Egypt. Thus the system requirements for the platform would be similar to those required for Egypt
% \subsubsection{Testing Environment}
 %\textcolor{green}{Note: I can't remember who told me this - but putting 'ideal' anywhere can very quickly sound like a cop-out. }
% Using real world environmental conditions, an environment can be developed for the aid of platform and equipment testing to allow for ease of test replication. The \sout{ideal}\textcolor{green}{targeted} operating environment \sout{includes} \textcolor{green}{is identified by}: 
% \begin{itemize}
 %\item Sandy flat ground 
 %\item Low mineral content
 %\item No metallic contamination \textcolor{red}{Is this right? I think we need to test for a lot of things. We do need to eventually test for metallic clutter - RP}
% \item No large buried rocks.
 %\end{itemize}
\subsection{Performance Specifications}
\seclabel{performance}
 Knowing the basic operating environment and chosen scenario of operations, basic performance specifications can be defined. These specifications include the platform, environment and sensors.
\subsubsection{Platform}
\seclabel{platform}
The platform is required to be able to perform certain tasks to ensure that the goals are completed. In order to do this it needs to satisfy some basic specifications and requirements which are highlighted below: 
\begin{itemize}
\item The platform must be able to travel forwards and maintain a certain speed while following a straight course. Reverse as well as left and right turning are also required for navigation purposes.
 \item The platform should be able to carry a payload of 100 kg. The MD and GPR with their respective control boxes have an estimated gross weight of 80 kg. This allows 20 kg for the necessary control units and computers, as well as any unforeseen weight additions or contributions. 
\item The platform should be able to be easily modified for controlling remotely as well as the attachment of sensors. Easy attachment of sensors would ensure that there are no unnecessary delays in the project due to design issues. Ideally, a platform which has the ability to have off the shelf supports attached to it would be ideal as minimal modifications would be required. This would result in more time for the testing of the system. 
\item The platform should be easily transportable with minimum disassembly required allowing for ease of transport from one location to another. Thus, it should be able to fit on the back of a sout{U}ute or small truck with minimal disassembly. 
\item Operate at a speed of 5 km/h as this is the current operational speed defined by the DSTG.
\item Come to a complete stop from operational speed in 1 \sout{meter}\textcolor{green}{metre [watch your Americanisations people]}.
\end{itemize}
\subsubsection{Environment}
\seclabel{environment}
\textbf{\textcolor{green}{I think this section needs to be changed to reflect that we haven't designed anything yet. We're imposing limits here on what we will design to, and therefore place limits on our deliverables. These restrictions here aren't for anyone but us. Needs a tense change IMO}}

The platform should be designed to operate under certain environmental conditions. These operating conditions are highlighted below:
\begin{itemize}
\item Terrain: The platform should be able to operate over dry, loose sand or gravel with sparse vegetation. Ideal terrains will be roads but testing on open fields with minimal obstacles and vegetation will also be conducted.
\item Gradient: The platform should be able to operate over a maximum slope angle of 15 degrees
\item Obstacles: The platform should be able to operate over sparse vegetation with some large obstacles such as a tree or wall
\item Moisture: The platform should be able to operate low moisture enviorments with no rain or mud
\end{itemize}
\subsection{Sensors}
\seclabel{sensors} %needs to be reworked
The sensor parameters for landmine detection that the project aims to comply to are listed below:
\begin{itemize}
\item Detection Depth: 300mm 
\item Detection speed: Real time detection at operational speed
\item Detection rate: Maximise detection rate while minimising false positives
\item Detection halo: 1 meter % this isn't possible with our sensors, maybe like 1m?
\end{itemize}

\section{Project Scope}
\seclabel{projectScope}
\textcolor{red}{This section will need to include: What the soil types are, what our operating conditions are, what is the methodology, etc. NOT a scenario of operations.} 
\textcolor{green}{MAZ COMMENT: involve your aims. talk about the scope of method as well.}
%Justification: A brief statement regarding the business need your project addresses. (A more detailed discussion of the justification for the project appears in the project charter.)
%Product scope description: The characteristics of the products, services, and/or results your project will produce.
%Acceptance criteria: The conditions that must be met before project deliverables are accepted.
%Deliverables: The products, services, and/or results your project will produce (also referred to as objectives).
%Project Exclusions: Statements about what the project will not accomplish or produce.
%Constraints: Restrictions that limit what you can achieve, how and when you can achieve it, and how much achieving it can cost.
%Assumptions: Statements about how you will address uncertain information as you conceive, plan, and perform your project.

The scope of this project is the automation and modification of an existing platform for landmine detection. Upon completion of the project as highlighted in \secref{primary}, automation of an existing platform is completed through an operator device, modification of the platform to carry multiple sensors, signal processing of GPR and metal detector data for landmine detection, and transmission of the state of the platform, landmine locations using GPS, and percentage probability of a landmine.

The automation of an existing platform is to assist with unmanned operations for landmine detection. The platform will be completed with the necessary framework for autonomous operation through basic remote control and transmission to an operator device. Both navigation and control systems will be included, such as, GPS, relevant platform sensors, actuators and micro controllers. These will be implemented using relevant navigation and control algorithms for the platform to travel in a straight line and turning at specified angles (\secref{navigationandautomation}). However, traversing autonomously within a specified area and to way-points will be outside of this scope. Autonomous operation of the platform is limited to the operations specified in \secref{performance}. 

An existing platform will be modified to accommodate for GPR and MD sensors through the design and manufacture of a sensor mount (\secref{sensormount}). The attachment will be designed according to platform and sensor dimensions, weight and operation limitation. The limitations to mount both MD and GPR sensors is the range of operation between them and the material used for the sensor mount.

The landmine detection scope of the project includes the autonomous detection and classification of subsurface objects based on statistical probability. The object is identified with a percentage likelihood that it is a landmine. Various signal processing methods for landmine detection using both GPR and MD sensors will be considered, including background and image processing and statistical (fuzzy logic) algorithms (\secref{splitreview}). These algorithms will assume both specific operating environments (as discussed in \secref{operatingEnvironments}) and subsurface electromagnetic properties. The operating environment for this project is limited to the environment conditions specified in \secref{environment}. These specifications are the minimum testing conditions required in order to achieve this objective. Physical marking and automatic removal of the landmines are not included within the scope of the project. 

The platform operation conditions, landmine location coordinates and percentage probability will be transmitted to an operator device. This is implemented through software development in conjunction with the remote control of the platform. The software device will not include live video feed of the current operations and mission data as this is outside of the project scope. Development of this software is limited to open source and minimal commercial software development packages (\textcolor{red}{insert section reference on concept design}).%\secref{}

\textcolor{red}{INSERT REFERENCE TO NEXT SECTIONS HERE?}
 
\end{document}